\chapter{Proofs}

\endinput

\begin{proof}
  We prove the claim \eqref{eq:invariant}
  by induction over $t = 1, \dotsc, d$.
  For $t = 1$, the right-hand side (RHS) of~\eqref{eq:invariant} equals
  $\{\ell \in \NN_0 \mid \ell \le n\}$, which is just $L^{(1)}$.
  
  For the induction step $t - 1 \to t$, we assume that
  \begin{equation}
  \label{eq:indhyp}
  \begin{aligned}
  L^{(t-1)}
  &= \{\vec{\ell'} \in \NN^{t-1} \mid
  \norm{\vec{\ell'}}_1 \le n+t-2\} \;\cup\\
  &\hphantom{=\;\;}
  \{\vec{\ell'} \in \NN_0^{t-1} \setminus \NN^{t-1} \mid
  \norm{\vec{\ell'}}_1 + N_{\vec{\ell'}} \le n+t-1-b\} \cup
  \{\vec{0}\}.
  \end{aligned}
  \end{equation}
  First, we show that every inserted level $\vec{\ell}$ in the inner loop
  can be found on the right-hand side of~\eqref{eq:invariant}.
  If $\vec{\ell} := (\vec{\ell'}, 0)$
  is inserted for some $\vec{\ell'} \in L^{(t-1)}$,
  then we have $\norm{\vec{\ell'}}_1 + (N_{\vec{\ell'}} + 1) \le n + t - b$ or
  $N_{\vec{\ell'}} = t - 1$ by the if clause.
  In the first case, we have
  \begin{equation*}
  \norm{\vec{\ell}}_1 + N_{\vec{\ell}}
  = \norm{\vec{\ell'}}_1 + (N_{\vec{\ell'}} + 1)
  \le n + t - b,
  \end{equation*}
  and in the second case, it holds $\vec{\ell} = \vec{0}$.
  by the if clause.
  In either case, $\vec{\ell}$ is contained in the RHS of~%
  \eqref{eq:invariant}.
  
  If $\vec{\ell} := (\vec{\ell'}, \ell_t)$ is inserted
  for some $\vec{\ell'} \in L^{(t-1)}$ and
  $1 \le \ell_t \le u_{\vec{\ell'}}$, then we have,
  depending on whether $N_{\vec{\ell'}} > 0$ or not, two cases:
  \begin{enumerate}
    \item
    If $N_{\vec{\ell'}} > 0$, then $\vec{\ell} \in \NN_0^t \setminus \NN^t$
    and
    \begin{align*}
    \norm{\vec{\ell}}_1 + N_{\vec{\ell}}
    %&= \norm{\vec{\ell'}}_1 + \ell_t + N_{\vec{\ell'}}\\
    &\le \norm{\vec{\ell'}}_1 + u_{\vec{\ell'}} + N_{\vec{\ell'}}\\
    &= \norm{\vec{\ell'}}_1 + \left(n + t - b -
    (\norm{\vec{\ell'}}_1 + N_{\vec{\ell'}})\right) + N_{\vec{\ell'}}\\
    &= n + t - b,
    \end{align*}
    i.e., $\vec{\ell}$ is contained in the second set of the RHS of~%
    \eqref{eq:invariant}.
    
    \item
    If $N_{\vec{\ell'}} = 0$, then $\vec{\ell} \in \NN^t$ and
    \begin{align*}
    \norm{\vec{\ell}}_1 + N_{\vec{\ell}}
    %&= \norm{\vec{\ell'}}_1 + \ell_t + N_{\vec{\ell'}}\\
    &\le \norm{\vec{\ell'}}_1 + u_{\vec{\ell'}} + N_{\vec{\ell'}}\\
    &= \norm{\vec{\ell'}}_1 +
    \left(n + t - 1 - \norm{\vec{\ell'}}_1\right)\\
    &= n + t - 1,
    \end{align*}
    i.e., $\vec{\ell}$ is contained in the first set of the RHS of~%
    \eqref{eq:invariant}.
  \end{enumerate}
  Therefore, all levels, which the algorithm inserts into $L^{(t)}$,
  can be found on the RHS of \eqref{eq:invariant}.
  
  It remains to prove that all levels on the RHS of~\eqref{eq:invariant}
  are inserted by the algorithm into $L^{(t)}$ eventually.
  First, we take some $\vec{\ell} = (\vec{\ell'}, \ell_t)$
  in the first set of the RHS,
  i.e., $\norm{\vec{\ell}}_1 \le n + t - 1$ and $\ell_t > 0$.
  Note that $\vec{\ell'}$ will be encountered in the loop 2b), as
  \begin{equation*}
  \vec{\ell'} \in \NN^{t-1},\quad
  \norm{\vec{\ell'}}_1
  = \norm{\vec{\ell}}_1 - \ell_t
  \le (n+t-1) - 1
  = n + t - 2
  \end{equation*}
  implies $\vec{\ell'} \in L^{(t-1)}$ by the induction
  hypothesis~\eqref{eq:indhyp}.
  Since $N_{\vec{\ell'}} = 0$ and
  \begin{equation*}
  \ell_t
  = \norm{\vec{\ell}}_1 - \norm{\vec{\ell'}}_1
  \le n+t-1 - \norm{\vec{\ell'}}_1
  = u_{\vec{\ell'}},
  \end{equation*}
  this $\vec{\ell}$ is inserted into $L^{(t)}$ during the innermost loop.
  
  Now, we take $\vec{\ell} = (\vec{\ell'}, \ell_t)$
  in the second set of the RHS, i.e.,
  $\norm{\vec{\ell}}_1 + N_{\vec{\ell}} \le n+t-b$ and
  $N_{\vec{\ell}} > 0$.
  Again, there are two cases:
  \begin{enumerate}
    \item
    $\ell_t > 0$:
    This implies $N_{\vec{\ell'}} = N_{\vec{\ell}} > 0$ and
    $\vec{\ell'}$ will be reached in the loop 2b) since
    \begin{equation*}
    \vec{\ell'} \in \NN_0^{t-1} \setminus \NN^{t-1},\quad
    \norm{\vec{\ell'}}_1 + N_{\vec{\ell'}}
    = (\norm{\vec{\ell}}_1 - \ell_t) + N_{\vec{\ell}}
    \le n+t-b-\ell_t
    \le n+t-1-b
    \end{equation*}
    implies $\vec{\ell'} \in L^{(t-1)}$ by the induction
    hypothesis~\eqref{eq:indhyp}.
    Also,
    \begin{equation*}
    \ell_t
    = \norm{\vec{\ell}}_1 - \norm{\vec{\ell'}}_1
    \le (n+t-b - N_{\vec{\ell}}) - \norm{\vec{\ell'}}_1
    = n+t-b - (\norm{\vec{\ell'}}_1 + N_{\vec{\ell'}})
    = u_{\vec{\ell'}},
    \end{equation*}
    i.e., $\vec{\ell}$ gets added to $L^{(t)}$
    by the innermost loop.
    
    \item
    $\ell_t = 0$:
    This implies
    $\norm{\vec{\ell'}}_1 = \norm{\vec{\ell}}_1$ and
    $N_{\vec{\ell'}} = N_{\vec{\ell}} - 1$.
    $\vec{\ell'}$ will be reached in the loop 2b) since
    \begin{equation*}
    \vec{\ell'} \in \NN_0^{t-1} \setminus \NN^{t-1},\quad
    \norm{\vec{\ell'}}_1 + N_{\vec{\ell'}}
    = \norm{\vec{\ell}}_1 + (N_{\vec{\ell}} - 1)
    \le n + t - 1 - b,
    \end{equation*}
    if $N_{\vec{\ell'}} > 0$ and
    \begin{equation*}
    \vec{\ell'} \in \NN^{t-1},\quad
    \norm{\vec{\ell'}}_1
    = \norm{\vec{\ell}}_1
    \le n + t - b - N_{\vec{\ell}}
    \le n + t - 2,
    \end{equation*}
    if $N_{\vec{\ell'}} = 0$
    (due to $N_{\vec{\ell}} = 1$ in this sub-case and $b \ge 1$).
    In both of these sub-cases,
    we have $\vec{\ell'} \in L^{(t-1)}$ by the induction
    hypothesis~\eqref{eq:indhyp}. Also,
    \begin{equation*}
    \norm{\vec{\ell'}}_1 + (N_{\vec{\ell'}} + 1)
    = \norm{\vec{\ell}}_1 + N_{\vec{\ell}}
    \le n + t - b.
    \end{equation*}
    i.e., $\vec{\ell}$ gets added to $L^{(t)}$
    by the if clause.
  \end{enumerate}
  
  Finally, we take $\vec{\ell} = (\vec{0}, 0) \in \NN_0^t$
  in the third set of the RHS.
  This level is appended to $L^{(t)}$, because
  $\vec{\ell'} = \vec{0} \in \NN_0^{t-1}$ is in $L^{(t-1)}$ by 
  induction hypothesis~\eqref{eq:indhyp} and $N_{\vec{\ell'}} = t - 1$
  (i.e., the second statement in the if clause is fulfilled).
\end{proof}
